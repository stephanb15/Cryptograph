\documentclass{article}
\usepackage[utf8]{inputenc}

\title{Programmierprojekt - Verschlüsselungsalgorithmen\\Gruppe 1 -  Pachinger Pia}
\author{Jakob Lanser a11806538@unet.univie.ac.at\\Saifullah Totakhel a11713253@unet.univie.ac.at\\Stephan Bornberg a01506156@unet.univie.ac.at}

\begin{document}

\maketitle

Es werden einfache Algorithmen verfasst, die einen deutlich verständlichen "Klartext" in einen nicht mehr verständlichen "Geheimtext" ("Chiffrat") konvertieren.\\
Jeder Algorithmus beschreibt eine unterschiedliche Chiffre, mit deren Hilfe ein vom Benutzer eingegebener Text verschlüsselt wird.\\
Folgende Chiffren sind für das Projekt vorgesehen :\\
1. Transpositionschiffre - Zeichen werden nach einem vorgegebenem Schema umsortiert.\\
2. Substitutionschiffre (oder allgemeiner Blockschiffre) - Zeichen (bzw. Blöcke von Zeichen) werden durch Zeichen (bzw. Blöcke von Zeichen) eines Chiffrentextes ersetzt.\\
3. Mehrfachverschlüsselungen - Endliche Kompositionen aus Verschlüsselungsgfunktionen (z.B. eine Komposition aus Transpositionschiffren und Substitutionschiffre).

Zu den Verschlüsselungsfunktionen sollen zudem Entschlüsselungsfunktionen erstellt werden (die Umkehrfunktionen der oberen Bijektionen). Das Programm soll für abgestimmte Schlüssel zweier Instanzen desgleichen, eine Verschlüsselte Kommunikation ermöglichen. (Der Chiffretext kann dabei mit bereits vorhandenen Protokollen von einem User an den anderen verschickt werden)

4. Gegebenenfalls wäre im Zuge des Projetes auch eine Entschlüsselung ohne bekannter Schlüssel durch einfache Algoritmen (z.B. brute-force ähnliche Herangehensweise) um den Schlüssel zu finden, vorzusehen (unter der Voraussetzung, dass in den Verschlüsselten Text Information enthalten ist, welche für einen Verbraucher sin machen würde. 

\end{document}
